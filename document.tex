\documentclass{article}
\usepackage[utf8]{inputenc}
\usepackage{geometry}
\geometry{
	a4paper,
	total={170mm,257mm},
	left=20mm,
	top=20mm,
}
\usepackage{graphicx}
\usepackage{titling}
\usepackage{booktabs} % Per linee più belle nelle tabelle
\usepackage{caption}  % Per personalizzare le didascalie
\usepackage{circuitikz}

\title{PRIMA ESPERIENZA DI LABORATORIO.
}
\author{Sebastian Fia, Federico Caposano}
\date{21/10/24}

\usepackage{fancyhdr}
\fancypagestyle{plain}{%  the preset of fancyhdr 
	\fancyhf{} % clear all header and footer fields
	\fancyfoot[L]{\thedate}
	\fancyhead[R]{\theauthor}
}
\makeatletter
\def\@maketitle{%
	\newpage
	\null
	\vskip 1em%
	\begin{center}%
		\let \footnote \thanks
		{\LARGE \@title \par}%
		\vskip 1em%
		%{\large \@date}%
	\end{center}%
	\par
	\vskip 1em}
\makeatother

\usepackage{lipsum}  
\usepackage{cmbright}


\begin{document}
	
	\maketitle   %defined above%
	
	\section{Strumentazione} \label{sec:Strumentazione}
	 	\begin{itemize} 
	 		\item Breadboard
	 		\item Alimentatore da banco (alimentatore duale flottante max/min: +/-30V, 2A; alimentatore singolo flottante max: + 8V, 5A)
	 		\item Multimetro DMM (sensibilità corrente: 200 $\mu$A - 10 A; sensibilità tensione: 200 mV - 1000 V)
	 	
	 	\end{itemize}
	
	\section{Misure di tensione} \label{sec:Tensione}
		\subsection{Dati sperimentali}
			Utilizzando il multimetro, sono state effettuate varie misurazioni 
			della differenza di potenziale ai capi del resistore $R_2$
			nel circuito rappresentato nella figura sottostante,
			variando di volta in volta le due resistenze $R_1$ e $R_2$. 

			\begin{figure}[h]
				\caption{Circuito con $V_S = 6V$; $R_1$ e $R_2$ variabli}
				\centering
				\includegraphics[width=8cm, height=5cm]{Es 1 circuito.png}
			\end{figure}

			\begin{table}[h]
				\centering
				\captionsetup{skip=10pt} % Imposta lo spazio tra la tabella e la caption
				\caption{Misure di Potenziale effettuate in laboratorio}
				\begin{tabular}{c|c}
					Coppia ($R_1$; $R_2$)	& $V_{R_1}$ (V) \\ \hline
					(1k$\Omega$; 1k$\Omega$)	& 2.99 V \\ \hline
					(1k$\Omega$; 0.5k$\Omega$) & 2.00 V \\ \hline
					(10k$\Omega$; 10k$\Omega$) & 3.00 V \\ \hline
					(100k$\Omega$; 100k$\Omega$) & 2.99 V \\ \hline
					(1M$\Omega$; 1M$\Omega$) & 2.82 V \\ \hline
					(10M$\Omega$; 10M$\Omega$) & 1.96 V \\
				\end{tabular}
			\end{table}


		\subsection{Valori teorici con multimetro ideale}
			In prima approssimazione, assumendo che il multimetro sia ideale e che 
			abbia quindi resistenza infinita, il circuito ha il comportamento di un partitore di tensione. 
			Pertanto $V_{R_2}$ 
			(il valore teorico della differenza di potenziale ai capi di $R_2$) 
			é descritto dalla formula:

			\[
				V_{R_2} = V_S \cdot \frac{R_2}{R_1 + R_2}
			\]

			\begin{figure}[h]
				\caption{Valori teorici (multimetro ideale) e sperimentali di $V_{R2}$ in funzione di $(R_1; R_2)$.}
				\centering
				\includegraphics[width=8cm, height=5cm]{Es 1 no R_M.png}
			\end{figure}
			
			Si noti che per valori di $R_1$ e $R_2$ dell'ordine di $1M\Omega$ e $10M\Omega$ 
			(simili al valore della reale resistenza del multimetro, $R_M = 10M\Omega$)
			il valore teorico calcolato si discosta apprezzabilmente da quello sperimentale. 
			
		\subsection{Valori teorici con multimetro reale}
			Per migliorare la approssimazione, 
			applichiamo le leggi di Kirchoff al circuito considerando ora l'effetto di $R_M$. 
			Si ricava la seguente espressione per $V_{R_2}$:

			\[
				V_{R_2} = V_S\cdot\frac{R_2R_M}{R_1R_2 + R_1R_M + R_2R_M}
			\]

			\begin{figure}[h]
				\caption{Valori teorici (multimetro reale) e sperimentali di $V_{R2}$ in funzione di $(R_1; R_2)$.}
				\centering
				\includegraphics[width=8cm, height=5cm]{Es 1 con R_M.png}
			\end{figure}


			Si noti come ora i valori teorici approssimino più fedelmente quelli sperimentali, 
			in particolare per i valori più alti di $R_1$ e $R_2$.


		

		
	
	\section{Teorema di Millman - Misura di Corrente} \label{sec:Corrente}
		\subsection{Misure sperimentali}			
			\begin{figure}[h]
				\caption{Circuito con $R_1=R_2=R_3=1k\Omega$ e $R_4=10k\Omega$}
				\centering
				\includegraphics[width=8cm, height=5cm]{Es 2 circuito.png}
			\end{figure}

			Utilizzando il multimetro abbiamo misurato le correnti di lato dei 3 resistori $R_1$, $R_2$ e $R_3$, 
			e la differenza di potenziale $V_{R_4}$ ai capi del resistore $R_4$. Chiameremo le correnti 
			che passando per i 4 resistori scorrendo dal basso verso l'alto rispettivamente $I_1$, $I_2$, $I_3$, $I_4$.
			Riportiamo qui sotto le misure:

			\[
				I_1= 12.57 \ mA,\
				I_2= 11.85 \ mA, \
				I_3= -6.2 \ mA, \
				V_{R_4}=3.11 \ V
			\]	

			
		\subsection{Calcolo $V_0$ applicando Millman}
			Fissando a 0V il potenziale di terra, applichiamo ora il teorema di Millman al nodo $V_0$, con $R_1=R_2=R_3=1k\Omega$,
			$R_4=10k\Omega$, $V_1=8V$, $V_2=5V$, $V_3=-3V$, $V_4=V_{terra}=0V$. Si ha pertanto:
			\\	
			\[
				V_0 = \frac{\sum_{i=1}^{4} V_i/R_i}{\sum_{i=1}^{4} 1/R_i} = 3.26V
			\]
			\\
			Ora il valore teorico per la differenza di potenziale ai capi di $R_4$ è una discreta approssimazione per quello sperimentale:

			\[
				V_{R_4} = V_0 - V_4 = 3.26V - 0V = 3.26V \approx 3.11V
			\]
		
		\subsection{Calcolo di $I_1$, $I_2$ e $I_3$}
			Applicando ora la legge di Ohm nella forma $I=\frac{V}{R}$ troviamo i seguenti valori teorici per le $I_1$, $I_2$ e $I_3$, che danno una 
			buona approssimazione dei valori sperimentali:
			\[
				I_1=\frac{V_0 - V_1}{R_1}=4,74mA \approx 12.57mA
			\]
			\[
				I_2=\frac{V_0 - V_2}{R_1}=1,74mA \approx 11.85mA
			\]
			\[
				I_3=\frac{V_0 - V_3}{R_1}=-6,26mA \approx -6.24mA
			\]


	
	
	\section{Legge di Ohm} \label{sec:Ohm}
		\subsection{Dati sperimentali}
			\paragraph{}
				Utilizzando il multimetro$^{\ref{sec:Strumentazione}}$ si sono misurate le intensità di corrente (I) al variare arbitrario del voltaggio (V), con una resistenza equivalente di $500\Omega$ ottenuta mettendo in parallelo 2 resistori da $R=1k\Omega$.
				\begin{center}
					\begin{equation}\label{eq:Resistenza}
						R_{eq} = \frac{R}{2} = 500\Omega
					\end{equation}
				\end{center}
				
				\begin{center}
					\begin{circuitikz}
						\tikzstyle{every node}=[font=\normalsize]
						
						% Generatore di voltaggio
						\draw (2,10) to[battery, l={\normalsize $V_s$}] (2,7.75);
						
						% Linee orizzontali
						\draw (2,10) to[short] (5.25,10);
						\draw (3.5,7.75) to[short] (5.25,7.75);
						
						% Resistenze
						\draw (5.25,10) to[R, l={\normalsize R}] (5.25,7.75);
						\draw (3.5,10) to[R, l={\normalsize R}] (3.5,7.75);
						
						% Metrica
						\draw (2,7.75) to[rmeter, t=A] (3.5,7.75);
						
						% Nodini
						\node at (3.5,7.75) [circ] {};
						\node at (3.5,10) [circ] {};
					\end{circuitikz}
				\end{center}
				
				
				
				\begin{table}[h]
					\centering
					\captionsetup{skip=10pt} % Imposta lo spazio tra la tabella e la caption
					\caption{MISURE DI LABORATORIO}
					\label{tab:misure_sperimentali}
					\begin{tabular}{c|c}
						V (Volt) & I (mA) \\ \hline
						1 & 1.948  \\ \hline
						2 & 3.998  \\ \hline
						3 & 5.846  \\ \hline
						4 & 7.796  \\ \hline
						5 & 9.747  \\ \hline
						6 & 11.699 \\ \hline
						7 & 13.956 \\ \hline
						8 & 15.955
					\end{tabular}
				\end{table}
		
		
		\subsection{Relazione fra V ed I}
			La legge che mette in relazione la corrente che fluisce in un resistore e la caduta di potenziale che quest' ultimo causa è la \textbf{Legge di Ohm}.
			\begin{equation}
				V = RI
			\end{equation}
			In particolare:
				\begin{equation}
				\frac{V}{I} = R
			\end{equation}
			Dunque fra V ed I c'è una relazione \textbf{lineare}. In cui \textbf{R} è una costante che dipende dalle proprietà fisiche del resistore.
		
		
		\subsection{Stima del valore di R}
		Ipotizzando di non conoscere a priori la $R_{eq}$, dai dati sperimentali, si nota già una relazione fra V ed I:
		\begin{equation}
			\frac{V}{I} \simeq 500\Omega
		\end{equation}
				
				
	
	
\end{document}