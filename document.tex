\documentclass{article}
\usepackage[utf8]{inputenc}
\usepackage{geometry}
\geometry{
	a4paper,
	total={170mm,257mm},
	left=20mm,
	top=20mm,
}
\usepackage{graphicx}
\usepackage{titling}

\title{PRIMA ESPERIENZA DI LABORATORIO.
}
\author{Sebastian Fia, Federico Caposano}
\date{21/10/24}

\usepackage{fancyhdr}
\fancypagestyle{plain}{%  the preset of fancyhdr 
	\fancyhf{} % clear all header and footer fields
	\fancyfoot[L]{\thedate}
	\fancyhead[R]{\theauthor}
}
\makeatletter
\def\@maketitle{%
	\newpage
	\null
	\vskip 1em%
	\begin{center}%
		\let \footnote \thanks
		{\LARGE \@title \par}%
		\vskip 1em%
		%{\large \@date}%
	\end{center}%
	\par
	\vskip 1em}
\makeatother

\usepackage{lipsum}  
\usepackage{cmbright}


\begin{document}
	
	\maketitle   %defined above%
	
	\section{Strumentazione} \label{sec:Strumentazione}
	 	\begin{itemize} 
	 		\item Breadboard
	 		\item Alimentatore da banco (alimentatore duale flottante +/-30V, 2A; alimentatore singolo flottante + 8V, 5A)
	 		\item Multimetro DMM (sensibilità corrente: 200 $\mu$A - 10 A; sensibilità tensione: 200 mV - 1000 V)
	 	\end{itemize}
	
	\section{Misure di tensione} \label{sec:Tensione}
	\section{Misure di Corrente} \label{sec:Corrente}
	
	
	
	\section{Legge di Ohm} \label{sec:Ohm}
	\subsection{Dati sperimentali}
	\subsection{Relazione fra I e V}
	\subsection{Stima del valore di R}
	
\end{document}